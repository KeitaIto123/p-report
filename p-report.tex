\documentclass[twocolumn]{preport}
\usepackage[dvipdfmx]{graphicx}
\graphicspath{{figs/}}

\title{2018年度 中間試問 要旨 \\
空陸両用マルチリンクロータ飛行ロボットの自律移動操作行動に関する研究}
\author{稲葉・岡田研究室 指導教員 稲葉雅幸教授 \\
東京大学 工学部機械情報工学科 4 年 03170267 伊藤慶太}

\begin{document}

\pagestyle{empty}
\maketitle
\thispagestyle{empty}
\sloppy

\section{研究の背景・目的}

近年、いわゆるドローンと称される回転翼を複数持つマルチロータ型飛行ロボットの研究[1][2][3]が盛んに行われており、空撮[4]や災害救助[5]などその応用は多岐に渡る。これらの応用例の中で、本研究では物体の運搬[5][6]に目を向けた。飛行ロボットによる物体運搬の特徴は、ロボットの移動が水面や起伏の激しさなどの地表面の環境に依存しないため、移動可能領域が格段に向上する点である。また、上空からの広い視野により網羅的に物体認識を行うことが可能[7]である。その一方で、空中ではロボットの自重を支えるための揚力が必要となり、地上で物体を運搬をする場合に比べて余分にエネルギーが必要となる欠点が存在する。
そこで、本研究ではエネルギー効率が良くないという飛行ロボットを用いた物体運搬における問題に対して、マニピュレータ能力を持つ2次元変形ロボット"hydrus"が空と陸の両方の移動方法をとるというアプローチにより取り組む。物体の認識を上空から障害物に邪魔されることなく網羅的に行い、物体の運搬を周りの環境に応じて空陸の移動手段を切り替えて行うことによりエネルギー消費を抑える。

本稿はプログレスレポートのテンプレートである\cite{Sakai}.

本稿における「、」や「。」は、\verb|make pub|を実行することで、「,」や「.」に変更される。

図は\figref{nowprinting}や\tabref{sample}として参照する.

\begin{figure}[tbh]
 \begin{center}
  \begin{minipage}{0.3\columnwidth}
   \includegraphics[width=\columnwidth]{nowprinting.eps}
   \caption{eps図の参考例}
  \end{minipage}
  \hspace{0.15\columnwidth}
  \begin{minipage}{0.3\columnwidth}
   \includegraphics[width=\columnwidth]{dj.jpg}
   \caption{jpg図の参考例}
  \end{minipage}
  \label{figure:nowprinting}
 \end{center}
\end{figure}

\begin{table}[tbh]
 \begin{center}
  \begin{tabular}{|l|r|} \hline
  A1 & B1 \\
  A2 & B2 \\ \hline
  \end{tabular}
  \caption{図の参考例}
  \label{table:sample}
 \end{center}
\end{table}

\section{本研究のシステム構成}
本研究では
\begin{enumerate}
\item 目標物体の認識
\item 目標物体の把持
\item 目的地への物体運搬
\end{enumerate}
を組み合わせたシステムの構築を目指す。


\subsection{目標物体の認識}


\subsection{目標物体の把持}

\subsection{目的地への物体運搬}

\section{考察とまとめ}



\bibliographystyle{junsrt}
\bibliography{p-report}

\end{document}

