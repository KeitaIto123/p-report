\documentclass[twocolumn]{preport}
\usepackage[dvipdfmx]{graphicx}
\graphicspath{{figs/}}

\title{2018年度 中間試問 要旨 \\
空陸両用マルチリンクロータ飛行ロボットの自律移動操作行動に関する研究}
\author{稲葉・岡田研究室 指導教員 稲葉雅幸教授 \\
東京大学 工学部機械情報工学科 4 年 03170267 伊藤慶太}

\begin{document}

\pagestyle{empty}
\maketitle
\thispagestyle{empty}
\sloppy

\section{はじめに}

近年、いわゆるドローンと称される回転翼を複数持つマルチロータ型飛行ロボットの研究[1][2][3]が盛んに行われており、実際に社会の中で役立っている例も挙げられる。例えば空撮[4]や人が立ち入れないような場所の点検[5]などである。これらの用途の中で、本研究では物体の運搬に目を向けた。
飛行ロボットによる物体の運搬は、有名な例を
本研究では、空中と陸上の2つの移動手段をとることが可能な飛行ロボットを用いて、自律移動ロボットによる物体の運搬という課題に取り組む。この研究の背景には、陸上のみを移動可能なロボットと空中のみを移動可能なロボットによる物体運搬における問題がある。前者は、陸上に存在する障害物によりロボットが移動と物体の認識が制限されるというものだ。後者は、空中ではロボットの自重を支えるためのエネルギーが常に必要となり長時間の飛行が難しいというものだ。この問題に対して、空陸両用マルチリンクロータ飛行ロボットを用いることにより、物体の認識を上空から障害物に邪魔されることなく網羅的に行い、物体の運搬時は周りの環境に応じて陸空の移動手段を切り替えてエネルギー消費を抑える、というアプローチを取ることができる。

本稿はプログレスレポートのテンプレートである\cite{Sakai}.

本稿における「、」や「。」は、\verb|make pub|を実行することで、「,」や「.」に変更される。

図は\figref{nowprinting}や\tabref{sample}として参照する.

\begin{figure}[tbh]
 \begin{center}
  \begin{minipage}{0.3\columnwidth}
   \includegraphics[width=\columnwidth]{nowprinting.eps}
   \caption{eps図の参考例}
  \end{minipage}
  \hspace{0.15\columnwidth}
  \begin{minipage}{0.3\columnwidth}
   \includegraphics[width=\columnwidth]{dj.jpg}
   \caption{jpg図の参考例}
  \end{minipage}
  \label{figure:nowprinting}
 \end{center}
\end{figure}

\begin{table}[tbh]
 \begin{center}
  \begin{tabular}{|l|r|} \hline
  A1 & B1 \\
  A2 & B2 \\ \hline
  \end{tabular}
  \caption{図の参考例}
  \label{table:sample}
 \end{center}
\end{table}

\section{研究概要}
\section{おわりに}
\section{結論と今後の指針}
\section{おわりに}


\bibliographystyle{junsrt}
\bibliography{p-report}

\end{document}

