\documentclass[twocolumn]{preport}
\usepackage[dvipdfmx]{graphicx}
\graphicspath{{figs/}}

\title{2018年度 中間試問 要旨 \\
空陸両用マルチリンクロータ飛行ロボットの自律移動操作行動に関する研究}
\author{稲葉・岡田研究室 指導教員 稲葉雅幸教授 \\
東京大学 工学部機械情報工学科 4 年 03170267 伊藤慶太}

\begin{document}

\pagestyle{empty}
\maketitle
\thispagestyle{empty}
\sloppy

\section{研究の背景・目的}
近年、いわゆるドローンと称される回転翼を複数持つマルチロータ型飛行ロボットの研究\cite{Sakai}[1][2][3]が盛んに行われており、空撮[4]や災害救助[5]などその応用は多岐に渡る。これらの応用例の中で、本研究では物体の運搬[5][6]に目を向けた。飛行ロボットによる物体運搬の特徴は、ロボットの移動が水面や起伏の激しさなどの地表面の環境に依存しないため、移動可能領域が格段に向上する点である。また、上空からの広い視野により網羅的に物体認識を行うことが可能[7]である。その一方で、空中ではロボットの自重を支えるための揚力が必要となり、地上で物体を運搬をする場合に比べて余分にエネルギーが必要となる欠点が存在する。

そこで、本研究ではエネルギー効率が良くないという飛行ロボットを用いた物体運搬における問題に対して、マニピュレータ能力を持つ2次元変形ロボット"hydrus"が空と陸の両方の移動方法をとるというアプローチにより取り組む。


図は\figref{nowprinting}や\tabref{sample}として参照する.

\begin{figure}[tbh]
 \begin{center}
  \begin{minipage}{0.3\columnwidth}
   \includegraphics[width=\columnwidth]{nowprinting.eps}
   \caption{eps図の参考例}
  \end{minipage}
  \hspace{0.15\columnwidth}
  \begin{minipage}{0.3\columnwidth}
   \includegraphics[width=\columnwidth]{dj.jpg}
   \caption{jpg図の参考例}
  \end{minipage}
  \label{figure:nowprinting}
 \end{center}
\end{figure}

\begin{table}[tbh]
 \begin{center}
  \begin{tabular}{|l|r|} \hline
  A1 & B1 \\
  A2 & B2 \\ \hline
  \end{tabular}
  \caption{図の参考例}
  \label{table:sample}
 \end{center}
\end{table}

\section{本研究のシステム構成}
本研究では
\begin{enumerate}
\item 目標物体の認識
\item 目標物体の把持
\item 目的地への物体運搬
\end{enumerate}
を組み合わせたシステムの構築を目指す。
物体の認識を上空から障害物に邪魔されることなく網羅的に行い、物体の運搬を周りの環境に応じて空陸の移動手段を切り替えて行うことによりエネルギー消費を抑える。
目標物体の認識は、上空から障害物に邪魔されることなく、広い視野により網羅的に行う。

\section{現在までに行ったこと}
\subsection{目標物体の認識・把持}
シュミレータ上で赤い円柱の物体の認識・把持ができることを確かめた。
具体的な方法を説明すると、hydrusに下向きに取り付けられたカメラの画像からhsi filterを用いて特定色を抽出し、そこから輪郭を抽出する。目標物体である円柱の大きさは既知として、目標物体からhydrusまでの高さすなわちhydrusの地上からの高さから円柱の高さを引いた高さの情報とから、カメラの画像に写る大きさを推定し、抽出した輪郭とおよそ一致しているかどうかフィルタリングをした上で、その輪郭の中心点の座標を取得する。あとはこの座標に向けて、目標物体の周りを囲むように着地し、関節角度を変えることで目標物体を把持する。

\subsection{目的地への物体運搬}
現在までに本研究の肝となるhydrusを用いた地上での移動方法の検討・検証を行った。

まず、hydrusを地上で移動させるにあたって、一般的なマルチロータ型飛行ロボットは上空でプロペラと同じ平面上を移動する際に機体を少し傾けることによりプロペラの推力の水平成分を生み出し移動している点に注目し、機体が傾いた状態で地面に接地するように改良を施すことを考えた。これは、先端にボールキャスターの付いた長さの異なる足を数カ所機体に取り付けることで実現した。このとき、試しにプロペラに一定の推力を与えてみたところ、hydrusは地上を走行し、この方法で問題ないことを確かめた。

次に、どのような制御方法により目的地までhydrusが地上を走行していくかを考えた。モデル式を立てて制御することも検討したが、
動きを前進・右折・左折の3パターンに絞る
それぞれの動きは4つのプロペラに一定の推力(回転数)を与えることで実現
この推力は事前に人が調整を行ったもの
目的地に対する姿勢と距離を下記のループで制御


\section{考察とまとめ}
わかったこと
Hydrusが地上を走行可能であること
上空移動で必要な推力の50%で移動可能

今後の課題
地上での移動に関する制御パラメータの調整の自動化
必要なパラメータを抽出し、強化学習を行う
物体を把持後の地上での物体の運搬方法の検討
現状では物体把持後に地上走行をすると、物体が地面と擦る
足の配置・長さを工夫することで解決を図る
地上の環境に応じた移動方式の選択
搭載カメラにより障害物を認識



\bibliographystyle{junsrt}
\bibliography{p-report}

\end{document}

